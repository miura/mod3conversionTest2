\section{Step 3: FISH spots detection}

Again we will perform a sequence of image processing steps manually and then include them in a macro. The starting point is to have the channels split and the nuclei stored in the ROI manager (if you lose the information at any time you can sequentially launch the macros from Steps 1 and 2 on the original hyperstack to get to this point).

\subsection{Workflow}
In this section we will segment all the spots in the three FISH channels. To detect the spots we use a similar prefiltering as for the segmentation of the nuclei (with a different smoothing scale).

\begin{enumerate}
    \item Apply Laplacian of Gaussian. We pre-filter the image of the first FISH channel with \ijmenu{[Plugins > Feature Extraction > FeatureJ >  FeatureJ Laplacian]}. The smoothing scale should be adjusted to the size of the spots!
    
\item Detect intensity regional minima with \ijmenu{[Process > Find Maxima...]}. You should tick \textbf{"Light background"} to detect minima and adjust \textbf{"Noise tolerance"} to optimize detection. Select \textbf{"Single Points"} as \textbf{"Output Type"} to create a binary mask with detected minima.
    
    \item Counting spots inside nuclei.
    In Step 2 we have already segmented the nuclei and stored them to the ROI manager. Now, we simply need to select the binary mask holding the detect spots and loop through the nuclei ROIs. Next, we count the number of pixels with intensity equal to 255 to retrieve the number of spots per nucleus.
    
    For each nucleus we will measure the statistics of the intensity with the macro function \textbf{getRawStatistics(nPixels, mean, min, max, std, histogram)} which among other returns the histogram of the pixel intensity inside the active selection.

\end{enumerate}
In Fig.~\ref{fig:nucleiLaplacian} we can observe the result of the LoG filter performed on the nuclei image. The shapes of the nuclei are nicely mimicked in rings of different intensities (Fig.~\ref{fig:NucAfterLaplacianLUT}), as it can be visualized by changing the LUT to a colored LUT (\ijmenu{[Image> LookUp Tables]} to get the list of available LUTs). Try to optimize the smoothing scale of FeatureJ Laplacian to obtain a result similar to \ref{fig:nucleiLaplacian}.

\item Set threshold and convert to mask. 
    Here, we will convert this image into a binary image by thresholding (see section on \textbf{thresholding} in Module 1). We need to set the bounds of the threshold to separate the nuclei from the background. To achieve this, we use the command:
   
    \ijmenu{[Image >  Adjust > Threshold...]}
    
    Make sure to un-tick "Dark Bakground" and "set background pixels to NaN" (appears as pop-up window when clicking "Apply"), to obtain a regular 8-bit binary image (mask). Thanks to the preprocessing the nuclei can now readily be segmented by thresholding the pixels with negative values (up to a small negative value) in the filtered image. 
    The results is a black and white image, in which all pixels that belong to an object have an intensity value of 255, while all pixels that belong to the background have an intensity value of 0. 
    
    \textbf{\underline{Note}} : After converting the image to binary ImageJ automatically applies (by default) a LUT inversion: The objects now appear black on a white background. See also  \url{http://imagej.nih.gov/ij/docs/guide/146-29.html#infobox:InvertedLutMask}

\item Fill holes
    \ijmenu{[Process >  Binary > Fill Holes]}.
    Some objects of the binary image might have holes (see Fig.~\ref{fig:NucFillHoles}). A hole is defined as a group of pixels belonging to the background (white pixels) surrounded by pixels belonging to the foreground (black pixels). Since we do not expect the nuclei to exhibit any hole, we use \ijmenu{[Process > Binary > Fill Holes]} to fill them in (see section \textbf{Morphology} in Module 1).
\textbf{sourcecode} : \href{http://www.example.com/contents}{code/solutions/module3\_01.ijm}

It is a good habit to add a preamble to a macro holding author, purpose, version, date and any helpful additional notes. Comments should also be added throughout the macro to summarize the aim of specific sub-sections (e.g. ``Initialization'', ``Erase the small particles''), especially if the sequence of commands is not straightforward to understand. These comments will often prove useful to people reading your code and to yourself when reading the code to modify it years after.

\subsection{Summary of tools used in Step 1}

Here is a summary of the main ImageJ tools used in this step:

\begin{itemize}
\item \textbf{Split Channels} \ijmenu{[Images > Color > Split Channels]}.\\
Split channels to get independent images for each fluorescence channel.

\item \textbf{Set Scale} \ijmenu{[Analyze > Set Scale...]}.\\
Allow to calibrate the pixel size

\item \textbf{Binary Options} \ijmenu{[Process>Binary> Options...]}.\\ 
Set the behavior of binary image related commands.

\item \textbf{Set Measurements} \ijmenu{[Analyze > Set Measurements...]}.\\ 
Set which features should be measured when calling \ijmenu{[Analyze > Measure]}.

\end{itemize}

\newpage
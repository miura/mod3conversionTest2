\section{Overview}

This module is divided into four separate steps. Each step has an enumerated workflow followed by an exercise. The result of each exercise is a macro. To ensure a smooth progress, the solution-macro for each exercise is provided and should be used as the starting point for consecutive exercises.

\subsection{Aim}

FISH (Fluorescent In-Situ Hybridization) is a complex gene staining technique with numerous variants \cite{volpi2008fish} where the quality of the staining depends on several physical parameters (e.g. level of DNA de-condensation). FISH aims at labeling DNA sequences specific to a certain gene (or chromosome) so that they appear as a "bright fluorescent spots" in fluorescent channels. We will write an ImageJ macro to process images from such a FISH assay: First to segment the spermatozoid nuclei from a DAPI staining, and then to classify the nuclei based on their chromosomal content (multiplicity of FISH spots in 3 different fluorescent channels).

\subsection{Introduction}

The automatic spermatozoids classification as proposed in \cite{molina2009fish} is a powerful mean to extract statistics of chromosomal anomalies (see Fig.~\ref{fig:intro}) on large sample datasets (typically >10'000 cells). It can also be used to drive a motorized microscope to perform an "intelligent" scan (\cite{tosi2012}): the classification is performed from a low resolution scan and a secondary scan (high resolution) is automatically triggered to only acquire the cells showing some specific anomalies.

\subsection{datasets}
The nuclei of the spermatozoids were DAPI stained and the chromosomes of interest (here X, Y and 18) were  stained by FISH. The fixed sample was scanned by a motorized stage microscope to tile the whole area of the sample. Several $z$ slices were recorded for each fluorescent channel (DAPI, aqua, orange, and green). The analysis will be performed on the z maximum intensity projection of the images (in each channel).
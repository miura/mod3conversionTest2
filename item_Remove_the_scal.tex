\item Remove the scale of the opened images by 
    \ijmenu{[Analyze > Set Scale\ldots]}. 
    Use the same settings as in Fig.~\ref{fig:init}:

\begin{itemize}
        \item\textbf{distance} and \textbf{known distance} to "0" and  
        \item\textbf{Pixel aspect ratio} to "1".  
        \item The \textbf{Unit length} should be "pixel". 
        \item  \textbf{Global} should be ticked so that the settings apply to all images (and subsequently opened images).

\end{itemize}

Alternatively, press the "click to remove scale" button. 
    
    \item Set Binary Options 
    \ijmenu{[Process > Binary > Options...]}.
    Make sure "Iterations" and "Count" are both set to 1 and that "Black background" and "Pad edges when eroding" are NOT ticked. EDM output should be set to overwrite. These settings correspond to ImageJ default settings and it is a good habit to always set them to predictable values at the beginning of a macro since they control the behavior of many commands that are commonly used. 
    
    \item Set Measurements 
    \ijmenu{[Analyze > Set Measurements...]}.

\begin{itemize}
        \item\textbf{Area,} 
        \item\textbf{Mean gray value,} 
        \item and \textbf{Shape Descriptors} 
        
        \item\textbf{Redirect} should be set to "None" 
        \item\textbf{Decimal places} should be 2 or greater.  
    
\end{itemize}

\end{enumerate}

\subsection{Exercise \arabic{exerciseCounter}}
\stepcounter{exerciseCounter}
Create a macro (from the recorder) that performs the steps 3 to 6 of the previous workflow.\footnote{If you are using OSX, it sometimes happens that copying from the command recorder and pasting it to the script editor does not work. In that case, try using right mouse click (or control-click) to copy recorded commands. If this still does not work, then click the ``Create'' button at the top-right corner.}. You can find the solution to the exercises as a complete macro in \textbf{code/solutions/module3\_01.ijm} .

\underline{\textbf{Note} }: We assume that the original hyperstack is already open when the macro is run.
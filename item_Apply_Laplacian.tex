\item Apply ``Laplacian of Gaussian'' (see Sec.~\textbf{Convolution} of Module~1 for an introduction to convolution filters ). We use a Laplacian of Gaussian (Log) filter as a preprocessing step to facilitate the segmentation (see section \ref{summary_of_tools_mod_3_step_2} for more information about the Log filter). In ImageJ this filter can be found in \ijmenu{[Plugins > Feature Extraction> FeatureJ > FeatureJ Laplacian]}.
    
    The first step of the filter is a Gaussian filter whose radius (smoothing scale) must be adjusted to the typical size of the nuclei. The next step is a Laplacian filter. When the smoothing scale is properly adjusted the nuclei appear as homogeneously dark and surrounded by a bright halo in the filtered image (see Fig.~\ref{fig:nucleiLaplacian}(b)). A rule of thumb is to set the smoothing scale to about half the expected radius of the objects. 
    
    Note, that the output of the filter is a 32 bit image since the intensity values can be negative or positive. FeatureJ Laplacian can also detect zero-crossings, where the intensity changes sign (close to a sharp intensity transition), but we will not use this feature here (leave unticked). Make sure that "Compute Laplacian image" is ticked.
    
    To better understand the advantage of using a Laplacian prefiltering, you can try to directly apply a threshold on the original nuclei image.